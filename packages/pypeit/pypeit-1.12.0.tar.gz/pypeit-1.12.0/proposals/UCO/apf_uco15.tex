\documentclass[11pt,preprint]{aastex}

\usepackage{latexsym}
\usepackage{fancybox}
\usepackage{graphicx}
\usepackage{amssymb}
\usepackage{color}
\usepackage{ulem}
\usepackage{float}
\usepackage{multicol}
\usepackage{enumitem}
%\usepackage{setspace}

\pretolerance=10000
\textwidth=6.4in
\textheight=8.9in
\voffset = 0.0in
\topmargin=0.0in
\headheight=0.00in
\hoffset = 0.0in
\headsep=0.20in
\oddsidemargin=0in
\evensidemargin=0in
\parindent=2em
\parskip=0.25ex
 
%\input{defs_xavier}
%\input{/u/xavier/bin/latex}
%\input{/u/xavier/bin/defs}

\newcommand{\npair}{50}
\def\ssp{\def\baselinestretch{1.0}\large\normalsize}
\newcommand{\oneskip}{\vskip \baselineskip}
\newcommand{\annrev}{ARA\&A}
\newcommand{\za}{z$_{ abs}$}
\newcommand{\zg}{z$_{ galaxy}$}
\newcommand{\zl}{$z_{emitter}$}
\newcommand{\zf}{z$_{df}$}
\newcommand{\avxi}{$\langle\xi\rangle$}
\newcommand{\omg}{$\Omega_{g}(z)$}
\newcommand{\tu}{$\tau( > M)$}
\newcommand{\mc}{$10^{12}M_{\odot}$}
\def \lnhi {$\log N_{HI}$}
\def \cmm  {cm$^{-2}$}
\def \cmmm {cm$^{-3}$}
\def \hfreq {$f_{\rm{HI}} (N,X)$}
\def \gamrate {$\Gamma_{912}$}
\def \fn {$f_{\rm{HI}} (N,X)$}
\def \lyaf {Ly--$\alpha$ forest}
\def \nqso  {seven}
\def \cmm  {\rm cm^{-2}}
\newcommand{\lya}{Ly$\alpha$}

\usepackage{multicol}

\newenvironment{my_itemize}{
\begin{itemize}
  \setlength{\itemsep}{1pt}
  \setlength{\parskip}{0pt}
  \setlength{\parsep}{0pt}}{\end{itemize}
}

\newenvironment{my_enumerate}{
\begin{enumerate}
  \setlength{\itemsep}{1pt}
  \setlength{\parskip}{0pt}
  \setlength{\parsep}{0pt}}{\end{enumerate}
}



\usepackage{geometry}
\geometry{letterpaper,tmargin=1.3in,bmargin=0.7in,lmargin=0.95in,rmargin=0.95in}
\usepackage[layout=modern]{advancedcoverpage}


\title{{\Large A Level~1 Data Reduction Pipeline for the APF}}

\presentedto{{\large \underline{2015 UCO Call for Instrumentation}} \\
\vskip 0.1in
{\it Submitted to the UCOAC}
}


\author{
{\bf PI:} Brad Holden (Astronomy \& Astrophysics; 831-495-xxxx; holden@ucolick.org) \\
\vskip 0.1in
{\bf UCSC Faculty Co-I's:}
J. Xavier Prochaska (Astronomy \& Astrophysics) \\
\vskip 0.1in
{\bf UCSC other Co-I's:} R. Cooke \\
\vskip 0.1in
%\includegraphics[width=5.5in]{../WhitePaper/title_fig2.pdf}
}

 
\begin{document}
\maketitle

	\pagestyle{myheadings}    % Go for customized headings
%\markright{Prochaska--U079 2008A Proposal---Quasars Probing Quasars}
	\markboth{\hfill Holden -- PyPit for APF (v1.0) \hfill}{\hfill
          Holden -- PyPit for APF \hfill}


\noindent 
{\bf Abstract:} \\
Blah

%%%%%%%%%%%%%%%%%%%%%%%%%%%%%%%%%%%%%%%%%%%%%%%%%%%%%%%%%%%%%
\vskip 0.2in

\noindent 
{\bf Summary:} \\
The Automated Planetary Finder (APF) facility is a UCO-owned and operated
observatory at Mt.\ Hamilton, developed for high-cadence relative velocity (RV)
observations to discover new planetary systems.
The facility has been in operation since 201X and $\approx 20\%$ of its observing
time is now competed for by the full UC astronomical community.
These community programs are a combination of RV work with the Iodine cell and
other non-Iodine projects that leverage the high-throughput and resolution
of the Levy spectrometer.  It is also expected that the [China] Observatory
will begin purchasing access to the APF.

With the growing user-base for the APF, there has been an increasing demand
for custom software to process the complex raw data frames into fully
calibrated spectra for analysis.  With the proposed funding, we would provide
a Level~1 data reduction pipeline (DRP) for the APF facility that offers
Th-Ar calibrated and fluxed spectra. [add one line]  
Further work would analyze Iodine cell observations for higher-precision
calibration.

%%%%%%%%%%%%%%%%%%%%%%%%%%%%%%%%
\vskip 0.2in

\noindent 
{\bf Technical:} \\
Over the past several months, Co-I Cooke has developed a Python-based
software package (PyPit) for the reduction of optical spectroscopy.
As of July 2015, this pipeline has been applied successfully to the 
Subaru/HDS and Keck/HIRES spectrometers.  The package is currently
under development for the Lick/Kast and the APF/Levy
spectrometers (without Iodine cell analysis).
It is the long-term goal of PyPit's contributors to release a
significant portion, if not all, of their software to the astronomical
community (via github).  The proposed funding would facilitate
this process, by enabling the development of custom routines
for the APF facility for (ideally) fully-automated processing
of the archival, raw data files.
%
It is our experience from years of developing DRPs and then releasing
these to the public (e.g.\ LowRedux, HIRedux) that substantial effort
is required to insure broad usage of the software.  Common
challenges include:
(i) insuring multi-platform compatability;
(ii) offering the software in a non-proprietary language;
(iii) developing comprehensive documentation and `cookbooks';
(iv) offering a suite of test cases.
We are now faced with these challenges as regards the Levy 
spectrometer and our PyPit package.  Although none of these are
insurmountable, all require significant person-hours of effort.
And, we emphasize that this effort comes with very little professional
gain for the lead contributors of PyPit.  This strongly motivates
the current proposal.

%%%%%%%%%%%%%%%%%%%%%%%%%
\vskip 0.2in

\noindent 
{\bf Work Plan:} \\
With \$XXk funding, we will provide the following deliverables
for the APF community, which comprises our Level~1 DRP:

{\small 
\begin{my_enumerate}
\item A Python-based GUI for organzing an ingested set of raw data
and calibration files (e.g.\ Figure~\ref{fig:GUI}).  
This GUI may also be used to launch execution
of the DRP (and control options).
\item Software to parse standard APF archival products.
\item Automated performance monitoring of the APF through analysis of 
spectrophotometric standard stars (e.g.\ http://ESI).
\item Extensive documentation of the APF-specific algorithms in PyPit.
\item A HOWTO cookbook for standard usage cases.
\item A suite of test cases for insuring that future releases of
PyPit continue to support the APF.
\item First-look spectra including S/N estimations.  
These may be archived on the APF site.
\item Insuring multi-platform funcionality, including but not limited
to Mac OS X 10.7 or greater, Linux Centos, and MS Windows [give version].
\end{my_enumerate}
}
To meet these deliverables in a one year time-scale, we are requesting
\$50k to cover XX months salary of PI Holden and 3 weeks of summer
salary for Co-I Prochaska (1 month at 75\% effort).
Co-I Cooke, the main contributor and author of PyPit, is fully funded
by a Hubble Fellowship.  His motivations to contributing to this project
are primarily scientific and to expand the user-base of his software.

PI Holden will lead this project and will distiribute his effort as 
follows: (i) XX months to deliverables \#X-X; (ii) etc.
Co-I Prochaska will concentrate on deliverables
\#4,5, and 7.

[NOT SURE WHERE TO PUT THIS;
Although we will not provide explicit software for RV analysis
with the Iodince cell, we will insure that the Level~1 products
seemlessly enable such analysis.]




%%%%%%%%%%%%%%%%%%%%%%%%%%
\clearpage

\begin{figure}
 \vskip -0.5in
\begin{center}
%\includegraphics[scale=0.55]{../WhitePaper/esi_footprint2.pdf}
\end{center}
 \caption{\footnotesize  
(left) Raw image of APF data
(right) Series of extracted echelle orders from PyPit
}\label{fig:redux}
%  \end{minipage}
\end{figure}

\begin{figure}
 \vskip -0.5in
\begin{center}
%\includegraphics[scale=0.59]{../WhitePaper/deployed_K2DM3.pdf}
\end{center}
 \caption{\footnotesize  
Example (notional) GUI
}\label{fig:GUI}
%  \end{minipage}
\end{figure}

\end{document}

\begin{document}



\end{document}