\Problem{El campanar de la Torrassa contraataca}


\Statement


\FigureR{width=5cm}{campanar} 
\emph{El Campanar de la Torrassa} és un campanar que es troba al barri de
Collblanc-La Torrassa de L'Hospitalet de Llobregat. Aquest campanar és ben
conegut per molestar dia i nit a tots els seus veïns amb les seves
campanades.

\medskip Recentment, el capellà d'aquesta església ha decidit iniciar
una col·lecta per tal de comprar una nova campana de ferro per posar
a dalt de tot del campanar. Els seus plans són de fer
sonar aquesta nova campana cada cop que les busques del rellotge se
solapen. Així, per exemple, la nova campana hauria de tocar a les 12:00
i cap a les 13:05. Segons el capellà, aquesta innovació doblarà el nombre de
fidels a missa (cosa que no li hauria de costar gaire perquè la seva església
és buida quasi sempre).

\medskip Abans de donar diners al capellà de la Torrassa, el bisbat
està interessat en saber el nombre de vegades que aquesta campana tocarà 
en un llarg periode de temps. En particular, el bisbat necessita un programa
que, donat un instant d'inici i una llargada de temps, compti el nombre
de cops que la nova campana sonarà en aquest periode de temps.
Cal tenir en compte que, a causa del retard inherent al moviment
del martell cap a la campana, la nova camapana sonarà trenta nou milisegons
després que les busques del rellotge coïncideixin.



\Input

L'entrada està formada per diferents jocs de proves, cadascun en una línia.
Cada joc de proves constisteix de tres enters $h$, $m$ i $t$. $h$ i $m$ 
representen l'instant d'inici 
($h$:$m$) i satisfan $0\le h\le 23$ i $0\le m\le 59$. $t$
representa la llargada, en minuts, del temps que cal mesurar i satisfà
$0\le t \le2^{30}$. 

\Output

Per a cada joc de proves $h$, $m$, $t$, 
la sortida ha d'incloure una línia amb un enter denotant
el nombre de cops que la nova campana sonarà començant a les 
$h$:$m$ i per un periode de $t$ minuts.


\Observation

Aquest problema va aparèixer a les semifinals del 3r Concurs de Programació de
la UPC.
