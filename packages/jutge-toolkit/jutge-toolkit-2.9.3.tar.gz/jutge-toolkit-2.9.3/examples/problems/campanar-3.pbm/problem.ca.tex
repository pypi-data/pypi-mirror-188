\Problem{El retorn del campanar de la Torrassa}


\Statement


\FigureR{width=5cm}{campanar} 
\emph{El Campanar de la Torrassa} és un campanar que es troba al districtre de
Collblanc-La Torrassa de L'Hospitalet de Llobregat. Aquest campanar és ben
conegut per molestar continuament a tots els seus veïns amb les seves
campanades.

\medskip Recentment, aquesta esgésia s'està quedant curta de diners, de forma que
el seu capellà ha decidit vendre una de les dues campanes del seu campanar
(concretament, la que tenia un so més greu). Amb una sola campana, ja no és
possible tocar les hores de la forma tradicional (vegeu problema \Link{LOOP}).
Per tant, el capellà ha inventat una nova forma de tocar les hores: A cada hora,
la campana sona tants cops com quina hora és (d'un a dotze); també,
la campana sona un cop als quarts, un cop als dos quarts i un cop també
als tres quarts. Per exemple, a les 19:00, la campana toca set cops.
El mateix passa a les 7:00. També, a les 19:15 la campana
toca un cop, i a les 19:30 i a les 19:45 també toca un cop.

\medskip

El capellà ha informat tot content als veïns que, amb el seu nou mètode,
les campanes toquen menys cops que abans. Per exemple, si algú es desperta
a la matinada, diguem a les 3:18, sentirà una campanada a les 3:30,
una altra a les 3:45 i quatre campanades a les 4:00. Podeu comprovar
que aquesta persona ha hagut d'esperar 42 minuts per poder saber quina
hora és exactament des que s'ha despertat, i que ha sentit sis campanades.
En canvi, amb el sistema tradicional, aquesta persona hauria sentit tretze
campanades fins saber quina hora és.

\medskip

Feu un programa que llegeixi una hora i escrigui el nombre de minuts
que algú ha d'esperar per saber exactament quina hora és i quantes
campanades ha de sentir mentrestant.

\Input

L'entrada conté diferents proves, una per línia. Cada prova conté
dos enters $h$ i $m$ que codefiquen l'hora de despertar-se ($h$:$m$) i
satisfan $0\le h\le 23$ i $0\le m\le 59$. 


\Output

Per a cada prova $h$:$m$, la sortida 
ha d'incloure una línia amb dos enters separats per un blanc.
El segon enter denota el nombre de minuts que algú ha d'esperar
(començant a $h$:$m$) per saber exactament quina hora és.
El primer enter denota el nombre de campanades que s'han sentit en
aquest interval de temps.


\Sample

\Observation

Aquest problema va aparèixer a les semifinals del 4t Concurs de Programació de
la UPC i conclou la Trilogia del Campanar de la Torrassa. En Jordi Petit ha
desmentit els rumors que diuen que està preparant una precuela.


